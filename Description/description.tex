\documentclass[a4paper,12pt,twoside]{article}

%% packages
\usepackage[french]{babel}
\usepackage[T1]{fontenc}
\usepackage[utf8]{inputenc}
\usepackage{amsmath,amssymb,mathrsfs} 
%\usepackage{import}
\usepackage{float, subfig, caption}
\usepackage{libertine}
\usepackage{fancybox}
\usepackage{graphicx}
\usepackage{authblk} % meilleur liste d'auteur
\usepackage{hyperref}
\hypersetup{colorlinks=true,linkcolor=blue,citecolor=red}
\urlstyle{same}
\usepackage[french]{minitoc}
%\usepackage[acronym,toc,automake]{glossaries} % acronym pour les acronymes et toc pour inclusion des glossary dans la table des matières. Ne doit pas etre inclue si \makeglossaries n'est pas invoqué!
\usepackage[left=2cm,right=2cm,top=2cm,bottom=2cm]{geometry}
%\usepackage{minted} % pour code avec coloration dans le texte
\usepackage{url} % Pour avoir de belles url
%% package ticks spécifiquement (pour les schémas) 
%\usepackage{tikz}%
%\usepackage[top=2cm,bottom=2cm]{geometry}
%% paramètres tikz
%\usetikzlibrary{patterns,decorations.pathreplacing,shapes.misc}
%\usetikzlibrary{calc}
%\tikzset{cross/.style={cross out, draw=black, minimum size=2*(#1-\pgflinewidth), inner sep=0pt, outer sep=0pt},
%default radius will be 1pt. 
%cross/.default={5pt}}
% \definecolor{Gray}{gray}{0.9}
% \definecolor{LightCyan}{rgb}{0.88,1,1}
% \newcommand{\RN}[1]{\textup{\uppercase\expandafter{\romannumeral#1}}}    

%% Macros
\newcommand{\incode}[1]{{\footnotesize\ttfamily #1}} % pour du pseudo-code simple
\newcommand{\rem}[2]{\noindent\underline{Remarque} : \textit{#1}\\ \indent #2}
\newcommand{\note}[1]{\noindent\underline{Note} : \\ \indent #1}
\newcommand{\defi}[2]{\noindent\underline{Définition} : \textbf{#1},\\ \indent #2}
\newcommand{\slide}[2]{\textbullet ~ Slide n°#1 : \indent #2}


%% Infos document
% Titre
\title{Description des fonctionnalités de l'EncyclopéBeer}
% auteurs
\author{Armel \and Clément}
% date de création du document
\date{\today}

%% Glossaire, acronymes et index
%\makeglossaries
%%% Glossaire
%\newglossaryentry{gl:us}
%{
%name={US}, 
%description={La description}
%}
%
%%% Acronyme
%\newglossaryentry{ac:us}
%{
%type = \acronymtype, name = {US}, 
%description = {User Story}, 
%first = {User Story (US)\glsadd{gl:us}}, 
%see = [Glossaire:]{gl:us}
%}

\begin{document}
%% Informations Générale
\maketitle
%\author{}
%\date{}
%% Organisation 
\tableofcontents%\addcontentsline{toc}{chapter}{Table des matières}
%\listoffigures\addcontentsline{toc}{chapter}{Table des figures}
%\listoftables\addcontentsline{toc}{chapter}{Liste des tableaux}

\section{Introduction}

L’EncyclopeBeer est une encyclopédie qui permet aux amateurs de bières de chercher, noter et garder en mémoire leurs bières favorites. \textcolor{blue}{Elle permet aussi de suivre sa consommation} \textcolor{red}{et d’obtenir des suggestions personnalisées en fonction de ses goûts.} \\

Pour les marchands elle permet de connaître le profil de ses clients : leur goûts et préférences afin d'adapter leurs marchandise, faire des suggestions personnalisées ou bien générales en fonction de la communauté (localisée proche du marchand). \\


\rem{Sur le code couleur employé dans ce document}{La couleur indique la priorité d'une fonctionnalité. 
\begin{itemize}
\item noir : fonctionnalité prioritaire,
\item \textcolor{blue}{fonctionnalité intéressante mais secondaire,}
\item \textcolor{red}{fonctionnalité à développer plus tard,}
\item \textcolor{green}{fonctionnalité opérationnelles.}
\end{itemize}
}

\section{Client Web}
L'utilisateur principale du client Web est le client, ou \textbf{Reader}.\\

\begin{itemize}
 \item Créer un compte, 
 \item Se connecter à son compte, 
 \item \textcolor{red}{Ajouter/Supprimer des contacts eux aussi inscrits.}
 \item Rechercher les produits pour : 
 	\begin{itemize}
 	 \item Noter un produit (avec commentaire écrit pour les clients \textit{certifiés}),
 	 \item Ajouter un produit aux favoris, 
 	 \item Ajouter un produit aux envies,
 	 \item \textcolor{red}{Suggérer un produit à un contact},
 	\end{itemize}
 \item Proposer un ajout de produit (informations minimums : nom, marque, type)
 \item Consulter : 
 	\begin{itemize}
 	 \item la liste des favoris, 
 	 \item la liste des envies, 
 	 \item \textcolor{blue}{l'historique des produits marqués comme consommés (quantités),} 
 	 \item \textcolor{red}{Les alertes du système sur sa consommation (code couleur pour différents états d'ébriété),}
 	 \item \textcolor{red}{Les recommandations des amis ou du système,}
 	\end{itemize}
 \item \textcolor{blue}{Ajouter un produit à sa consommation.}
 \item \textcolor{red}{Ajouter une consommation (non bière) en unité d'alcool ou degrés/volume à son historique de consommation.}
\end{itemize}

\section{Client lourd}

L'utilisateur principale est l'administrateur, ou \textbf{Admin}. \\

\begin{itemize}
 \item Ajouter un produit (directement ou via la liste de proposition des Readers,)
 \item Supprimer un produit, 
 \item Modifier un produit, 
 \item Gérer les utilisateurs (promotion des droits, modération des commentaires),
 \item Obtenir les statistiques détaillées d'un Readers \textcolor{blue}{à développer plus si historique de consommation},
 \item Obtenir les statistiques détaillées de l'ensemble (globales) des Readers\textcolor{blue}{à développer plus si historique de consommation}
 \item Obtenir les statistiques par produit (qui aime quoi, quel age, quel genre, quel localisation,...),
 \item \textcolor{red}{Analyser ces statistiques pour alimenter un système de recommandation de produit par le système.}
\end{itemize}
 
\section{Idées en vrac}
\note{Le code couleur ne s'applique plus içi}. \\

\begin{itemize}
 \item Un administrateur doit pouvoir élever les droits
 \item Un Reader \textit{certifié} devrait pouvoir obtenir certains des droits administrateur. (Ajouter un produit, modérer les commentaires, obtenir ses stats et pourquoi pas celle de ses contacts),
 \item Un Marchand devrait pouvoir indiquer la disponibilité d'un produit dans son commerce. 
 \item Un partenariat avec certain Marchand serait sympa (sa donnerait un ``coté lyon's club'' avec avantage client : Exemple promo sur le top 10 locale, etc...)
\end{itemize}

\section{Recommandations}
Sur l'ordre des choses à faire : \\

\begin{enumerate}
 \item Pensez la logique (les algos, les design pattern à utiliser), 
 \item Pensez aux problèmes techniques associés à chaque fonctionnalités,
 \item Pensez la base de donnée (en s'appuyant sur les Use Cases),
 \item Faire la base de données (avec les script pour CrUDe),
 \item Blinder le Diagramme de Classe,
 \item Coder les classes (et faire la doc en même temps dont la description de toutes les fonctions avec entrée, sortie, logique). Ne pas oublier de gérer toutes les exceptions (ex :  champ de saisie, problèmes mathématiques possibles)
 \item Coder l'interface (boutons, formulaires, tt ce qui a un champ de saisie, une fonction derrière un bouton),
 \item Pensez au (blinder le) design (Sketch),
 \item Appliquer le design(Application du Sketch et \textit{faire la peinture}),
 \item Tester : test unitaire pour chaque classe et méthode puis test d'intégration et enfin test de validation
\end{enumerate}

%% Glossaires
%\newpage
%\printglossary[type=\acronymtype]%\addcontentsline{toc}{chapter}{Acronymes}
%%\glsaddall% force l'apparition de tt les entrées du glossaire
%\glsaddallunused % meme chose que addall mais ne force pas la numéroattion dans la liste d'acronyme
%%\printunsrtglossary[type=main]
%\printglossary[type = main,nonumberlist]%\addcontentsline{toc}{chapter}{Glossaire}

\end{document}
